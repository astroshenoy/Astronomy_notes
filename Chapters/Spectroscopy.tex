\chapter{Spectroscopy}

 Among ion–ion correlations, \cite{Burchett_2015} find evidence for tight correlations between C II and Si II, C II and Si III, and C IV and Si IV, suggesting that these pairs of species arise in similar ionization conditions. However, the evidence for correlations decreases as the difference in ionization potential increases. When controlling for observational bias, they find only marginal evidence for a correlation (86.8\% likelihood) between the Doppler line width b(C IV) and column density N(C IV).


\section{Fitting emission lines}

One assumption people make while fitting emission lines is that the detected lines can be fitted simultaneously under assumption of local thermodynamic equilibrium (LTE). 

 Second, the effects of dielectronic recombination may contribute to enhancing the level populations even at large n.

 \section{Radio recombination lines}

 

 Radio recombination lines (RRLs) are commonly defined as radio spectral lines resulting from \textbf{transitions of high-n levels} of atoms, appearing after the \textbf{recombination of singly inoized ions and electrons} \href{https://ui.adsabs.harvard.edu/abs/2002ASSL..282.....G/abstract}{\textcolor{blue}{(Gordon \& Sorochenko 2002)}}\\
 \\
 \subsubsection{RRLs of ions with mass higher than Helium}
Searching for RRLs of ions heavier than helium towards the Sun has been unsuccessful (Berger \& Simon 1972); \href{https://link.springer.com/article/10.1134/S1063772922060038}{\textcolor{blue}{(Dravskikh \& Dravskikh 2022)}}.\\
\\
 \href{https://arxiv.org/pdf/2302.03398.pdf}{\textcolor{blue}{Liu Xunchuan+2023}} report the first detection of radio recombination lines (RRLs) of ions heavier than helium. In a highly sensitive multi-band (12–50 GHz) line survey toward Orion KL with the TianMa 65-m Radio Telescope (TMRT). 