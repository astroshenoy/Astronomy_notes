\chapter{Galaxies}

\section{Disc Galaxies}

\subsection{Barred Disc Galaxies}

Many disc galaxies host galactic bars, which exert time-dependent, non-axisymmetric forces that can alter the orbits of stars. \href{https://ui.adsabs.harvard.edu/abs/2023arXiv230201307F/abstract}{\textcolor{blue}{Filion+2023}}  find that the \textbf{bar induces both azimuth angle- and radius-dependent trends} in the median distance that stars have travelled to enter a given annulus. Angle-dependent trends are present at all radii they considered, and the \textbf{radius-dependent trends roughly divide the disc into three ‘zones’.} In the inner zone, stars generally originated
at larger radii and their orbits evolved inwards. Stars in the outer zone likely originated at smaller radii and their orbits evolved outwards. In the intermediate zone, there is no net inwards or outwards evolution of orbits. They comment on the possibility of using observed angle-dependent metallicity trends to learn about the initial metallicity gradient(s) and the radial re-arrangement that occurred in the disc.


\section{Stellar Mass}


\section{Stellar mass and velocity dipersion}
\href{https://iopscience.iop.org/article/10.3847/0004-637X/832/2/203/pdf}{Zahid+2016} :The stellar mass and velocity dispersion are governed by different physical processes, but both are intimately related to properties of the dark matter halo. \textbf{Stellar mass and velocity dispersion are strongly correlated, making it difficult to determine which of these two parameters is fundamental}. The stellar mass is an integral over the star formation history (SFH) and the end product of the complex baryonic processes governing galaxy formation and evolution. In contrast, the velocity dispersion is a measure of the stellar kinematics and is directly related to the gravitational potential of the system.



\section{Velocity Dispersion}

For spiral galaxies, the increase in velocity dispersion in population I stars is a gradual process which likely results from the random momentum exchanges, known as \textbf{dynamical friction}, between individual stars and large interstellar media (gas and dust clouds) with masses greater than 10$^5$ M$\odot$. Face-on spiral galaxies have a central $\sigma$ $\leq$ 90 km/s; slightly more if viewed edge-on.
\\

\textbf{Determining mass using H alpha}: The H-alpha line saturates (self-absorbs) relatively easily because hydrogen is the primary component of nebulae, so while it can indicate the shape and extent of the cloud, it cannot be used to accurately determine the cloud's mass. Instead, molecules such as carbon dioxide, carbon monoxide, formaldehyde, ammonia, or acetonitrile are typically used to determine the mass of a cloud.


