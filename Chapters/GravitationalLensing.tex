\chapter{Gravitational Lensing}


\section{What is gravitational lensing?}

\textbf{How do you estimate accurate redshift of a lensed galaxy?}
Does spectra shift when a galaxy is gravitationally lensed?
\\
\\
\href{https://ui.adsabs.harvard.edu/abs/2023arXiv230111264M/abstract}{Mainali et al. 2023} -- present new observations of sixteen bright (r = 19 − 21) gravitationally lensed galaxies at z $\simeq$ 1−3 selected from the CASSOWARY survey. In this paper, they investigate the rest-frame UV nebular line emission in our sample with the goal of understanding the factors that regulate strong CIII] emission. They compare the rest-optical line properties of high redshift galaxies with strong and weak CIII] emission, and find that systems with the strongest UV line emission tend to have young stellar populations and nebular gas that is moderately metal-poor and highly ionized, consistent with trends seen at low and high redshift. They find that \colorbox{yellow}{gas traced by the CIII] doublet likely probes higher densities than that traced by [OII] and [SII].} \\ Characterisation of the spectrally resolved Mg II emission line and several low ionization absorption lines suggests neutral gas around the young stars is likely optically thin, potentially facilitating the escape of ionizing radiation.


\section{Influence of lensing on the spectra : QSOs}

Recently Wandel, Peterson \& Malkan (1999) and Kaspi et al. (2000) using the reverberation technique discovered that the size of the \colorbox{yellow}{broad-line region (BLR; see Fig. \ref{fig:agn_schematic}) was smaller than that assumed} in the standard active galactic nucleus (AGN) model.