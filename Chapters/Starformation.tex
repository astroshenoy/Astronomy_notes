\chapter{Star formation in galaxies}

\section{Regulation of SF}

Processes like magnetic fields and turbulence regulate the star forming efficiency by acting against this collapse. But the most important mechanism that regulates collapse is self-regulation (Yan et al. 2023): when stars form in MCs, protostellar winds and massive OB stars destroy the clouds via ionization, heating by ultraviolet photons, stellar winds, and supernova blast waves. Although these processes can quench star formation locally, the shock waves can compress gas in another cloud, inducing star formation globally. These processes play an important role that regulates galaxy
formation and evolution.


For star formation to take place, gas and dust need to be sufficiently cold for gravity to overcome thermal pressure, and the ionisation fraction must be low enough to enable substantial decoupling between the gas and the Galactic magnetic field.



\href{https://arxiv.org/pdf/2303.05054.pdf}{This paper} used a spatially resolved SED to observe modestly suppressed star formation in the inner kiloparsec of the galaxy, which suggests that we are witnessing the early stages of inside-out quenching. They develop a parametric lens model to reconstruct the source-plane structure of dust imaged by the Atacama Large Millimeter/submillimeter Array, far-UV to optical light from Hubble, and near-IR imaging with 8 filters of JWST/NIRCam, as part of the Prime Extragalactic Areas for Reionization and Lensing Science (PEARLS) program.