
\chapter{Galaxy Mergers }


\section{Formation of bars}

There are three models of bar-formation considered in \href{https://arxiv.org/pdf/2006.14847.pdf}{Cavanagh et al. 2020}:
\begin{enumerate}
    \item Isolated model : Holh 1971 found that slow-growing non-axisymmetric disturbances eventually led to the formation of a central bar structure.
    \item Tidal model
    \item Merger model
\end{enumerate}

Mergers with \textbf{low mass ratios} and closely-aligned orientations are \textbf{more conducive} to bar formation than equal-mass mergers (\href{https://arxiv.org/pdf/2006.14847.pdf}{Cavanagh et al. 2020})

\section{Presence of bars in spirals}

"Around 30\% of all spiral galaxies are understood to be strongly barred (Sellwood \& Wilkinson 1993), while overall, more than 50\% of spiral galaxies typically contain features indicative of a central bar structure (Eskridge \& Frogel 1999). Bars are found to be even more prevalent when observed in infrared (Eskridge et al. 2000)." (\href{https://arxiv.org/pdf/2006.14847.pdf}{Cavanagh et al. 2020})